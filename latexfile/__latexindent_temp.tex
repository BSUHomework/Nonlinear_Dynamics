\documentclass{article}
% \usepackage{txfonts}
% \usepackage{graphicx}
%导言区
\usepackage{graphicx}
\usepackage{epstopdf}
\usepackage{float} 
\usepackage{subfigure}
\usepackage[]{caption2}
\begin{document}
Task 1: pages 36-40 ex. 2.2.1, 2.2.3, 2.2.8, 2.4.1, 2.4.3;

Task 2: pages 79-83 ex. 3.1.1, 3.2.3, 3.4.1, 3.4.5, 3.4.6;

Task 3: pages 141-144 ex. 5.1.7, 5.2.3, 5.2.5, 5.3.4;

Task 4: pages 182-184 ex. 6.3.1, 6.3.3, 6.4.2.

2.2 Analyze  the  following  equations  graphically. 
 In  each  case,  sketch  the  vector field on the real line, 
 find all the fixed points, classify their stability, and 
 sketch the graph of x(t) for different initial conditions. 
 Then try for a few minutes to obtain  the  analytical  solution
   for  x(t);  if  you  get  stuck,  don’t  try  for  too  long since 
   in several cases it’s impossible to solve the equation in closed form!

2.2.1 $ \dot x =4x^2-16 $



\begin{figure}[H]
\centering
\subfigure[2.2.1]{
\includegraphics[scale=0.7]{/Users/soliva/Desktop/Homework/Nonlinear_Dynamics/latexfile/221.pdf}
}
\end{figure}




2.2.3 $ \dot x =x-x^3 $

\begin{figure}[H]
\centering
\subfigure[2.2.3]{
\includegraphics[scale=0.7]{/Users/soliva/Desktop/Homework/Nonlinear_Dynamics/latexfile/223.pdf}}
\end{figure}

2.2.8 (Working backwards, from flows to equations) Given an equation , we know how to sketch the corresponding flow on the real line. Here you are asked to solve the opposite problem: For the phase portrait shown in Figure 1, find  an  equation  that  is  consistent  with  it.  (There  are  an  infinite  number  of correct answers—and wrong ones too.)

\includegraphics[scale=0.2]{/Users/soliva/Desktop/Homework/Nonlinear_Dynamics/latexfile/228question.png}
$$(x+1)^2x(x-2)$$
\includegraphics[scale=0.1]{/Users/soliva/Desktop/Homework/Nonlinear_Dynamics/latexfile/228.jpg}



2.4 Linear Stability Analysis
Use linear stability analysis to classify the fixed points of the following systems.
If linear stability analysis fails because f′(x*) = 0, use a graphical argument to

2.4.1 $\dot x =x(1-x)$
\begin{figure}[H]
\centering
\subfigure[2.4.1]{
\includegraphics[scale=0.7]{/Users/soliva/Desktop/Homework/Nonlinear_Dynamics/latexfile/241.pdf}}
\end{figure}


2.4.3 $\dot x =tan(x)$
\begin{figure}[H]
\centering
\subfigure[2.4.3]{
\includegraphics[scale=0.7]{/Users/soliva/Desktop/Homework/Nonlinear_Dynamics/latexfile/243.pdf}}
\end{figure}






3.1.1 $\dot x =1+rx+x^2$
\begin{figure}[H]
          \centering
          \subfigure[3.1.1]{
          \includegraphics[scale=0.7]{/Users/soliva/Desktop/Homework/Nonlinear_Dynamics/latexfile/411.pdf}}
          \end{figure}


3.2 Transcritical Bifurcation

For each of the following exercises, sketch all the qualitatively different vector fields that
occur as r is varied. Show that a transcritical bifurcation occurs at a critical value of r,
to be determined. Finally, sketch the bifurcation diagram of fixed points x* vs. r.

3.2.3 $\dot x = x − rx(1−x)$

\begin{figure}[H]
\centering
\subfigure[3.2.3]{
\includegraphics[scale=0.7]{/Users/soliva/Desktop/Homework/Nonlinear_Dynamics/latexfile/323.pdf}}
\end{figure}
3.4  Pitchfork Bifurcation
In the following exercises, sketch all the qualitatively different vector fields that occur as r is varied. Show that a pitchfork bifurcation occurs at a critical value of r (to be determined) and classify the bifurcation as supercritical or subcritical.
Finally, sketch the bifurcation diagram of x* vs. r.

3.4.1 $\dot x = rx-4x^3$
\begin{figure}[H]
\centering
\subfigure[3.4.1]{
\includegraphics[scale=0.7]{/Users/soliva/Desktop/Homework/Nonlinear_Dynamics/latexfile/341.pdf}}
\end{figure}


The  next  exercises  are  designed  to  test  your  ability  to  distinguish  
among  the various types of bifurcations—it’s easy to confuse them! In each case,
 find the values  of  r  at  which  bifurcations  occur,  and  classify  those  as
   saddle-node, transcritical, supercritical pitchfork, or subcritical pitchfork.
    Finally, sketch the bifurcation diagram of fixed points x * vs. r.

3.4.5 $\dot x = r-3x^2$

\begin{figure}[H]
\centering
\subfigure[3.4.5]{
\includegraphics[scale=0.7]{/Users/soliva/Desktop/Homework/Nonlinear_Dynamics/latexfile/345.pdf}}
\end{figure}

3.4.6 $\dot x = rx-x/(1+x)$

\begin{figure}[H]
\centering
\subfigure[3.4.6]{
\includegraphics[scale=0.7]{/Users/soliva/Desktop/Homework/Nonlinear_Dynamics/latexfile/346.pdf}}
\end{figure}
5

Sketch  the  vector  field  for  the  following  systems.  Indicate  the  length  and direction  of  the  vectors  with  reasonable  accuracy.  Sketch  some  typical trajectories.

5.1.7 $\dot x = x ,\dot y=x+y $

\begin{figure}[H]
\centering
\subfigure[5.1.7]{
\includegraphics[scale=0.7]{/Users/soliva/Desktop/Homework/Nonlinear_Dynamics/latexfile/517.pdf}}
\end{figure}


5.2.3 $\dot x = y ,\dot y=2x-3y $

\begin{figure}[H]
\centering
\subfigure[5.2.3]{
\includegraphics[scale=0.7]{/Users/soliva/Desktop/Homework/Nonlinear_Dynamics/latexfile/523.pdf}}
\end{figure}

5.2.5 $\dot x = 3x-4y ,\dot y=x-y $

\begin{figure}[H]
\centering
\subfigure[5.2.5]{
\includegraphics[scale=0.7]{/Users/soliva/Desktop/Homework/Nonlinear_Dynamics/latexfile/525.pdf}}
\end{figure}

5.3.4 $analyze \dot R =aJ ,\dot J =bR$

\begin{figure}[H]
  \centering
  \subfigure[5.3.4]{
  \includegraphics[scale=0.7]{/Users/soliva/Desktop/Homework/Nonlinear_Dynamics/latexfile/534.pdf}}
  \end{figure}

6.3
For each of the following systems, find the fixed points, classify them, sketch the neighboring trajectories, and try to fill in the rest of the phase portrait.

6.3.1 $\dot x =x-y ,\dot y =x^2 -4$

\begin{figure}[H]
  \centering
  \subfigure[6.3.1]{
  \includegraphics[scale=0.7]{/Users/soliva/Desktop/Homework/Nonlinear_Dynamics/latexfile/631.pdf}}
  \end{figure}

6.4.2 $\dot x = x(3-2x-y), \dot y = y(2-x-y)$

\begin{figure}[H]
  \centering
  \subfigure[6.4.2]{
  \includegraphics[scale=0.7]{/Users/soliva/Desktop/Homework/Nonlinear_Dynamics/latexfile/642.pdf}}
  \end{figure}




\end{document}