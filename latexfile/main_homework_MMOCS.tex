\documentclass{article}
% \usepackage{txfonts}
% \usepackage{graphicx}
%导言区
\usepackage{graphicx}
\usepackage{epstopdf}
\usepackage{float} 
\usepackage{subfigure}
\usepackage[]{caption2}
\begin{document}
Task 1: pages 36-40 ex. 2.2.1, 2.2.3, 2.2.8, 2.4.1, 2.4.3;

Task 2: pages 79-83 ex. 3.1.1, 3.2.3, 3.4.1, 3.4.5, 3.4.6;

Task 3: pages 141-144 ex. 5.1.7, 5.2.3, 5.2.5, 5.3.4;

Task 4: pages 182-184 ex. 6.3.1, 6.3.3, 6.4.2.


2.2.1 $ \dot x =4x^2-16 $



\begin{figure}[H]
\centering
\subfigure[2.2.1]{
\includegraphics[scale=0.7]{/Users/soliva/Desktop/Homework/Nonlinear_Dynamics/latexfile/221.pdf}
}
\end{figure}




2.2.3 $ \dot x =x-x^3 $

\begin{figure}[H]
\centering
\subfigure[2.2.3]{
\includegraphics[scale=0.7]{/Users/soliva/Desktop/Homework/Nonlinear_Dynamics/latexfile/223.pdf}}
\end{figure}

2.2.8 (Working backwards, from flows to equations) Given an equation , we know how to sketch the corresponding flow on the real line. Here you are asked to solve the opposite problem: For the phase portrait shown in Figure 1, find  an  equation  that  is  consistent  with  it.  (There  are  an  infinite  number  of correct answers—and wrong ones too.)

\includegraphics[scale=0.2]{/Users/soliva/Desktop/Homework/Nonlinear_Dynamics/latexfile/228question.png}
$$(x+1)^2x(x-2)$$
\includegraphics[scale=0.1]{/Users/soliva/Desktop/Homework/Nonlinear_Dynamics/latexfile/228.jpg}

2.4.1 $\dot x =x(1-x)$
\begin{figure}[H]
\centering
\subfigure[2.4.1]{
\includegraphics[scale=0.7]{/Users/soliva/Desktop/Homework/Nonlinear_Dynamics/latexfile/241.pdf}}
\end{figure}


2.4.3 $\dot x =tan(x)$
\begin{figure}[H]
\centering
\subfigure[2.4.3]{
\includegraphics[scale=0.7]{/Users/soliva/Desktop/Homework/Nonlinear_Dynamics/latexfile/243.pdf}}
\end{figure}






3.1.1 $\dot x =1+rx+x^2$
\begin{figure}[H]
          \centering
          \subfigure[3.1.1]{
          \includegraphics[scale=0.7]{/Users/soliva/Desktop/Homework/Nonlinear_Dynamics/latexfile/411.pdf}}
          \end{figure}


3.2 Transcritical Bifurcation
For each of the following exercises, sketch all the qualitatively different vector fields that
occur as r is varied. Show that a transcritical bifurcation occurs at a critical value of r,
to be determined. Finally, sketch the bifurcation diagram of fixed points x∗ vs. r.

3.2.3 $\dot x = x − rx(1−x)$

\begin{figure}[H]
          \centering
          \subfigure[3.2.3]{
          \includegraphics[scale=0.7]{/Users/soliva/Desktop/Homework/Nonlinear_Dynamics/latexfile/323.pdf}}
          \end{figure}


\end{document}